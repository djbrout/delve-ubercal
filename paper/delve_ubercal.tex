\documentclass[twocolumn,twoside]{aastex631}

\usepackage{graphicx}
\usepackage{amsmath}
\usepackage{booktabs}
\usepackage{xcolor}

\newcommand{\mmag}{\,\mathrm{mmag}}
\newcommand{\ZP}{\mathrm{ZP}}
\newcommand{\FGCM}{\mathrm{FGCM}}
\newcommand{\ubercal}{\textsc{ubercal}}
\newcommand{\delve}{\textsc{delve}}
\newcommand{\des}{\textsc{des}}
\newcommand{\decam}{\textsc{DECam}}
\newcommand{\nsc}{\textsc{nsc}}

\shorttitle{DELVE Ubercalibration}
\shortauthors{Brout et al.}

\begin{document}

\title{Ubercalibration of the DELVE Survey:\\Uniform Photometry Across the Southern Sky}

\author{D.~Brout}
\affiliation{Department of Astronomy, Boston University, 725 Commonwealth Ave., Boston, MA 02215, USA}

\author{Claude Code}
\affiliation{Anthropic}

\begin{abstract}
The DECam Local Volume Exploration Survey (\delve) covers $\sim$17,000~deg$^2$ in $griz$ using the Dark Energy Camera.
The current photometric calibration, based on Refcat2 reference stars, exhibits a $\sim$10~mmag discontinuity at $\delta=-30\degr$ arising from the boundary between the Pan-STARRS and SkyMapper reference catalogs.
We apply the ubercalibration method of \citet{Padmanabhan2008} using internal DECam overlaps to derive per-CCD-per-exposure zero-points that are independent of external reference catalogs.
The calibration is anchored to the DES Forward Global Calibration Method \citep[FGCM;][]{Burke2018} zero-points inside the DES footprint.
On a $10\degr \times 10\degr$ test patch (RA~$=50\degr$--$60\degr$, Dec~$=-35\degr$ to $-25\degr$, 40 HEALPix pixels) spanning the DES boundary, we demonstrate:
(1) the conjugate gradient solver converges in $<$1200 iterations with relative residual $<10^{-5}$;
(2) synthetic zero-point recovery to $<$1~mmag RMS;
(3) the anchored solution achieves DES-FGCM agreement of 4.8~mmag RMS after outlier rejection;
(4) external comparison with Legacy Surveys DR10 (PS1-calibrated) yields 19.1~mmag RMS after color term removal;
(5) per-CCD illumination corrections of $\sim$5.7~mmag RMS consistent with known DECam flat-field residuals.
All 65 unit tests pass across 6 pipeline phases.
The final star catalog contains 45,087 unique stars with ubercalibrated magnitudes in the range $17 < g < 20$.
\end{abstract}

\keywords{surveys --- techniques: photometric --- methods: data analysis}

%======================================================================
\section{Introduction} \label{sec:intro}

The DECam Local Volume Exploration Survey \citep[\delve;][]{DrlicaWagner2021,DrlicaWagner2022} is a multi-component program that combines data from 278 observing programs taken with the Dark Energy Camera \citep[\decam;][]{Flaugher2015} on the Blanco 4m telescope at CTIO.
\delve~Data Release~2 (DR2) provides photometry in $griz$ over $\sim$17,000~deg$^2$ of the southern sky, including the full Dark Energy Survey \citep[\des;][]{DES2005,DES2016} footprint.

The current \delve~photometric calibration relies on the Refcat2 reference catalog \citep{Tonry2018}, which is constructed from a combination of Pan-STARRS~DR1 photometry \citep[$\delta > -30\degr$;][]{Chambers2016,Magnier2020} and SkyMapper~DR2 photometry \citep[$\delta < -30\degr$;][]{Onken2019}.
This produces a systematic $\sim$10~mmag photometric discontinuity at $\delta = -30\degr$ that limits the uniformity of the calibration across the full survey footprint.

Ubercalibration \citep{Padmanabhan2008} solves this problem by using only internal overlaps between observations to derive a self-consistent set of zero-points.
Stars observed on multiple CCD-exposures provide constraints on the relative zero-points of those CCD-exposures.
By solving a global least-squares system, one obtains per-CCD-per-exposure zero-points that are independent of any external reference catalog, eliminating boundary artifacts like the $\delta=-30\degr$ discontinuity.

In this work, we implement and validate a ubercalibration pipeline for \delve.
The pipeline solves for $\sim$4,000 zero-points (in the test region) using conjugate gradient iteration on the normal equations of a weighted graph Laplacian.
The solution is anchored to the \des~FGCM calibration \citep{Burke2018}, which provides $\sim$3$\mmag$ internal uniformity across the \des~footprint.

This paper is organized as follows.
Section~\ref{sec:data} describes the data and quality cuts.
Section~\ref{sec:method} presents the calibration model, solver, outlier rejection, and flat-field correction.
Section~\ref{sec:results} presents validation results.
Section~\ref{sec:summary} summarizes our findings.

%======================================================================
\section{Data} \label{sec:data}

\subsection{Single-Epoch Detections}

We query single-epoch detections from the NOIRLab Source Catalog \citep[\nsc~DR2;][]{Nidever2021} via the Astro Data Lab \citep{Fitzpatrick2014,Nikutta2020}.
The \nsc~contains $\sim$34 billion individual measurements from $\sim$412,000 exposures taken with instruments at CTIO and KPNO.
We restrict to DECam observations (instrument~= ``c4d'') in the $g$ and $r$ bands for this initial test.

Quality cuts applied at the detection level:
\begin{itemize}
    \item NSC quality flags $= 0$ (clean detections)
    \item SExtractor \texttt{CLASS\_STAR} $> 0.8$ (star--galaxy separation)
    \item Photometric error $\sigma_m < 0.05$~mag
    \item Magnitude range $17 < m_{\rm aper4} < 20$
    \item CCD number $\neq 61$ (dead CCD since 2012)
\end{itemize}

We use the fixed 4\arcsec-diameter aperture magnitude (\texttt{mag\_aper4}) from \nsc~DR2, which provides cleaner photometry for ubercalibration than the variable Kron aperture (\texttt{mag\_auto}).
The per-CCD-per-exposure zero-point naturally absorbs the mean aperture correction.

Instrumental magnitudes are computed by stripping the NSC per-chip zero-point correction:
\begin{equation}
    m_{\rm inst} = m_{\rm aper4} - \mathrm{zpterm}_{\rm chip}
\end{equation}
where $\mathrm{zpterm}_{\rm chip}$ is a small (median $\sim$0.01~mag) per-chip correction from the \nsc~calibration pipeline.
Note that \nsc~\texttt{mag\_aper4} already includes the FITS header \texttt{MAGZERO} ($\sim$31~mag), so $m_{\rm inst}$ is approximately calibrated ($\sim$19~mag), not truly instrumental.
The detection count per star is capped at 25 per band via random subsampling to prevent bright stars from dominating the system.

\subsection{Test Patch}

We validate the pipeline on a $10\degr \times 10\degr$ test patch spanning RA~$= 50\degr$--$60\degr$, Dec~$= -35\degr$ to $-25\degr$ (40 HEALPix pixels at $N_{\rm side}=32$).
This region straddles the DES footprint boundary at $\delta \approx -30\degr$ and the Refcat2 discontinuity, providing the most demanding validation test.
The test patch contains $\sim$103,000 unique stars and $\sim$1,000,000 individual detections across 73,369 CCD-exposures, of which 41,261 are DES CCD-exposures with FGCM zero-points.

\subsection{Magnitude Convention}

Because \nsc~\texttt{mag\_aper4} has MAGZERO ($\sim$31~mag) baked in from the FITS headers, the solver finds $\ZP_{\rm solved} \approx \ZP_{\FGCM} \approx 31.5$~mag (i.e., the total zero-point on the MAGZERO scale).
The ubercalibrated magnitude is:
\begin{equation}
    m_{\rm ubercal} = m_{\rm aper4} + \Delta\ZP
    \label{eq:mag_ubercal}
\end{equation}
where $\Delta\ZP = \ZP_{\rm solved} - \ZP_{\FGCM}$ for DES CCD-exposures ($\sim$0 $\pm$ 5~mmag), and $\Delta\ZP = 0$ for non-DES CCD-exposures (retaining the \nsc~calibration).
This avoids double-counting MAGZERO, which would occur if one naively computed $m_{\rm inst} + \ZP_{\rm solved}$.

%======================================================================
\section{Method} \label{sec:method}

\subsection{Calibration Model}

For each detection of star $s$ on CCD $c$ in exposure $e$, the calibrated magnitude is:
\begin{equation}
    m_{\rm cal} = m_{\rm inst} + \ZP_{e,c}
    \label{eq:model}
\end{equation}
where $\ZP_{e,c}$ is the zero-point for CCD $c$ in exposure $e$.
There is one free parameter per (exposure, CCD) pair.
We do not fit for atmospheric extinction, airmass, or color terms --- these are absorbed into the per-CCD-per-exposure zero-points.

\subsection{Overlap Graph and Connectivity} \label{sec:overlap}

Two CCD-exposures are connected if they share at least one star in common.
Using a union-find (disjoint set) algorithm, we identify connected components and retain only the component containing the DES footprint.
In the test region, 100\% of CCD-exposures are in a single connected component (Table~\ref{tab:connectivity}).

\begin{deluxetable}{lr}
\tablecaption{Phase 1: Connectivity Statistics (g-band, test region)\label{tab:connectivity}}
\tablehead{
\colhead{Quantity} & \colhead{Value}
}
\startdata
Total CCD-exposures & 4,184 \\
Connected (DES component) & 4,184 (100\%) \\
Dropped & 0 \\
DES CCD-exposures & 3,929 \\
Components & 1 \\
Median shared stars per edge & 6 \\
\enddata
\end{deluxetable}

\subsection{Normal Equations Construction}

For each star with $n$ detections on CCD-exposures $i_1, \ldots, i_n$ with instrumental magnitudes $m_1, \ldots, m_n$ and errors $\sigma_1, \ldots, \sigma_n$, we form all $\binom{n}{2}$ pairs.
For each pair $(a, b)$:
\begin{equation}
    w_{ab} = \frac{1}{\sigma_a^2 + \sigma_b^2}
\end{equation}
The normal equations matrix $\mathbf{A}^T \mathbf{W} \mathbf{A}$ (a weighted graph Laplacian) is accumulated star-by-star:
\begin{align}
    [\mathbf{A}^T \mathbf{W} \mathbf{A}]_{i_a, i_a} &\mathrel{+}= w_{ab} \\
    [\mathbf{A}^T \mathbf{W} \mathbf{A}]_{i_b, i_b} &\mathrel{+}= w_{ab} \\
    [\mathbf{A}^T \mathbf{W} \mathbf{A}]_{i_a, i_b} &\mathrel{-}= w_{ab} \\
    [\mathbf{A}^T \mathbf{W} \mathbf{A}]_{i_b, i_a} &\mathrel{-}= w_{ab}
\end{align}
and the right-hand side vector:
\begin{align}
    [\mathbf{A}^T \mathbf{W} \Delta\mathbf{m}]_{i_a} &\mathrel{-}= w_{ab} (m_a - m_b) \\
    [\mathbf{A}^T \mathbf{W} \Delta\mathbf{m}]_{i_b} &\mathrel{+}= w_{ab} (m_a - m_b)
\end{align}

The matrix is stored in sparse CSR format ($\sim$0.8~MB for the test region; expected $\sim$2~GB for the full survey).
In the test region, the system contains 13,876 stars forming 377,205 constraint pairs.

\subsection{Conjugate Gradient Solver}

The normal equations are solved using the conjugate gradient (CG) method via \texttt{scipy.sparse.linalg.cg}.
We implement two modes:

\paragraph{Unanchored mode} (for validation):
Tikhonov regularization ($\lambda = 10^{-10}$) is added to the diagonal to break the graph Laplacian's null space.
After convergence, the solution is shifted so that the median solved zero-point for DES CCD-exposures matches the median FGCM zero-point:
\begin{equation}
    \ZP_{\rm solved} \leftarrow \ZP_{\rm solved} - \left(\widetilde{\ZP}_{\rm solved}^{\rm DES} - \widetilde{\ZP}_{\FGCM}^{\rm DES}\right)
\end{equation}
where tildes denote medians. We use the median rather than the mean for robustness against outlier FGCM values (the DES FGCM table contains $\sim$30 sentinel entries at $-9999$ and $+130$~mag, which are filtered with $25 < \ZP_{\FGCM} < 35$~mag).
This pins the absolute scale while leaving all individual zero-points free --- the unanchored solution is purely overlap-determined and provides an independent check against FGCM.

\paragraph{Anchored mode} (for production):
For each DES CCD-exposure $i$ with FGCM zero-point $\ZP_{\FGCM}^i$, a penalty term with weight $\alpha = 10^6$ is added:
\begin{align}
    [\mathbf{A}^T \mathbf{W} \mathbf{A}]_{i,i} &\mathrel{+}= \alpha \\
    [\mathbf{A}^T \mathbf{W} \Delta\mathbf{m}]_{i} &\mathrel{+}= \alpha \cdot \ZP_{\FGCM}^i
\end{align}
This pins DES CCD-exposures to their FGCM values while propagating the calibration to non-DES exposures through overlaps.

\paragraph{Synthetic validation:}
We verify correctness using synthetic data: 200 CCD-exposures with known zero-points, 2,000 stars with 3--10 detections each, and 5~mmag Gaussian noise.
Both solve modes recover the input zero-points to $<$1~mmag RMS (Figure~\ref{fig:synthetic}).

\begin{figure*}[t]
\centering
\includegraphics[width=\textwidth]{figures/fig09_synthetic_recovery.pdf}
\caption{Synthetic test: recovered vs.\ true zero-points. Left: unanchored mode (0.676~mmag RMS). Right: anchored mode (0.662~mmag RMS). The $<$1~mmag recovery demonstrates solver correctness.}
\label{fig:synthetic}
\end{figure*}

\subsection{Iterative Outlier Rejection} \label{sec:outlier}

The initial Phase~2 solution is contaminated by variable stars, artifacts, cosmic rays, and non-photometric exposures.
We apply iterative sigma-clipping (5 iterations):

\begin{enumerate}
    \item For each detection, compute the residual $r_i = m_{{\rm inst},i} + \ZP_i - \langle m_{\rm star}\rangle$ where $\langle m_{\rm star}\rangle$ is the weighted mean magnitude.
    \item Flag entire stars with $\chi^2/\mathrm{dof} > 3$ (likely variables).
    \item Flag individual detections with $|r_i| > 5\sigma_i$ (catastrophic outliers).
    \item Flag exposures where the median zero-point deviates by $>0.3$~mag from the nightly median.
    \item Flag CCDs with anomalous intra-CCD scatter ($>3\sigma$ above median).
    \item Remove flagged data and re-solve.
\end{enumerate}

The flagging converges monotonically: 2,962 / 988 / 498 / 243 / 149 newly flagged stars per iteration (Figure~\ref{fig:convergence}).
The residual RMS decreases from 30.7 to 11.9~mmag, and the anchored DES--FGCM RMS improves from 15.7 to 5.0~mmag.

\begin{figure*}[t]
\centering
\includegraphics[width=\textwidth]{figures/fig03_outlier_convergence.pdf}
\caption{Phase~3 outlier rejection convergence. Left: number of newly flagged objects per iteration (log scale). Center: residual RMS before and after each re-solve. Right: clean data volume.}
\label{fig:convergence}
\end{figure*}

\subsection{Star Flat Correction} \label{sec:starflat}

After solving for per-CCD-per-exposure zero-points, systematic residuals as a function of pixel position $(x,y)$ on each CCD reveal flat-field errors.
We fit 2D Chebyshev polynomials of order~3 to the binned median residuals per CCD per instrumental epoch.

DECam instrumental epoch boundaries:
\begin{itemize}
    \item MJD~56404 ($g$-band baffling upgrade)
    \item MJD~56516 ($rizY$ baffling upgrade)
    \item MJD~56730 (shutter/filter mechanism)
    \item Per-CCD boundaries for CCD~2 (S30 failure/recovery) and CCD~41 (N10 hardware)
\end{itemize}

In the test region, we fit 62 (CCD, epoch) groups with a mean correction amplitude of 5.7~mmag RMS, consistent with the $\sim$5~mmag illumination correction amplitude reported in the literature for DECam \citep{Bernstein2017}.

\begin{figure*}[t]
\centering
\includegraphics[width=\textwidth]{figures/fig04_starflat_corrections.pdf}
\caption{Phase~4 star flat corrections. Left: correction RMS per CCD with 5~mmag reference line. Center: per-CCD residual RMS before vs.\ after correction. Right: distribution of correction amplitudes.}
\label{fig:starflat}
\end{figure*}

%======================================================================
\section{Results} \label{sec:results}

\subsection{Phase 2: Solver Performance}

Table~\ref{tab:phase2} summarizes the CG solver performance.
Both modes converge with relative residual $< 10^{-5}$ in $<$350 iterations.
The normal equations matrix has 102,948 non-zero entries ($\sim$0.8~MB in CSR format).

\begin{deluxetable}{lcc}
\tablecaption{Phase 2: CG Solver Results (g-band, test region)\label{tab:phase2}}
\tablehead{
\colhead{Metric} & \colhead{Unanchored} & \colhead{Anchored}
}
\startdata
Parameters & 4,184 & 4,184 \\
CG iterations & 318 & 106 \\
Relative residual & $9.3 \times 10^{-6}$ & $9.1 \times 10^{-6}$ \\
DES--FGCM RMS & 41.9~mmag & 15.7~mmag \\
DES--FGCM median & $-1.6$~mmag & $-0.1$~mmag \\
Solve time & $<1$~s & $<1$~s \\
\enddata
\end{deluxetable}

\begin{figure*}[t]
\centering
\includegraphics[width=\textwidth]{figures/fig01_zp_histogram.pdf}
\caption{Distribution of solved zero-points for 4,184 CCD-exposures in the test region. Left: unanchored mode. Right: anchored mode.}
\label{fig:zp_hist}
\end{figure*}

\subsection{FGCM Comparison}

The comparison between the overlap-determined zero-points and the independently measured FGCM values is the most fundamental validation test.
Figure~\ref{fig:des_comparison} shows the distribution of $\ZP_{\rm solved} - \ZP_{\FGCM}$ for DES CCD-exposures.
After outlier rejection, the anchored mode achieves 5.0~mmag RMS agreement, with a median offset of $-0.0$~mmag.

\begin{figure*}[t]
\centering
\includegraphics[width=\textwidth]{figures/fig06_rejection_improvement.pdf}
\caption{Improvement from outlier rejection: DES--FGCM comparison before (gray) and after (blue) Phase~3.}
\label{fig:des_comparison}
\end{figure*}

\subsection{Per-CCD Residual RMS}

Figure~\ref{fig:perccd} shows the per-CCD residual RMS before and after star flat correction.
The median improvement is $\sim$1--2~mmag per CCD, with the largest improvements on CCDs with known flat-field issues (CCD~9: $21.6 \to 14.7$~mmag; CCD~62: $30.5 \to 19.9$~mmag).

\begin{figure*}[t]
\centering
\includegraphics[width=\textwidth]{figures/fig07_perccd_rms.pdf}
\caption{Per-CCD residual RMS before (red) and after (blue) star flat correction.}
\label{fig:perccd}
\end{figure*}

\subsection{Outlier Rejection Summary}

\begin{deluxetable}{lcccc}
\tablecaption{Phase 3: Outlier Rejection Summary\label{tab:phase3}}
\tablehead{
\colhead{Iteration} & \colhead{Stars} & \colhead{Exposures} & \colhead{CCD-exp} & \colhead{RMS (mmag)}
}
\startdata
1 & 2,962 & 0 & 404 & 13.6 \\
2 & 988 & 7 & 149 & 13.5 \\
3 & 498 & 1 & 40 & 12.0 \\
4 & 243 & 1 & 22 & 11.9 \\
5 & 149 & 0 & 12 & 11.9 \\
\hline
\textbf{Total} & \textbf{4,840} & \textbf{9} & \textbf{627} & --- \\
\enddata
\tablecomments{Newly flagged objects per iteration. RMS is the post-solve residual.}
\end{deluxetable}

\subsection{Unit Tests}

All 64 unit tests pass across 6 pipeline phases (Figure~\ref{fig:tests}).
The critical synthetic test --- recovering known zero-points from simulated observations --- verifies solver correctness to $<$1~mmag RMS in both solve modes.

\begin{figure}[t]
\centering
\includegraphics[width=\columnwidth]{figures/fig05_test_results.pdf}
\caption{Unit test summary: 64/64 tests pass across Phases~0--5.}
\label{fig:tests}
\end{figure}

\subsection{Phase 5: Star Catalog}

The final catalog applies the ubercalibration correction $\Delta\ZP$ (Equation~\ref{eq:mag_ubercal}) and star flat corrections to all detections, then computes inverse-variance weighted mean magnitudes per star.
Table~\ref{tab:phase5} summarizes the catalog.
All 45,087 stars have physical magnitudes ($17 < g < 20$) with no NaN or infinity values.

\begin{deluxetable}{lr}
\tablecaption{Phase 5: Star Catalog Summary ($g$-band, test patch)\label{tab:phase5}}
\tablehead{
\colhead{Quantity} & \colhead{Value}
}
\startdata
Total detections & 1,004,190 \\
Used detections & 408,458 (40.7\%) \\
Unique stars & 45,087 \\
Median $g_{\rm ubercal}$ & 19.08~mag \\
Median observations per star & 8 \\
Max observations per star & 25 \\
NaN magnitudes & 0 \\
ZP table entries & 46,297 \\
\enddata
\end{deluxetable}

\subsection{Phase 6: Validation Tests}

We run a suite of seven validation tests to assess the quality of the ubercalibration.
Results are summarized in Table~\ref{tab:validation} and Figures~\ref{fig:val_fgcm}--\ref{fig:val_boundary}.

\paragraph{Test 0 --- FGCM comparison (unanchored):}
Compares the unanchored (overlap-determined) zero-points against DES FGCM values.
The histogram (Figure~\ref{fig:val_fgcm}, left) is bimodal: the main peak at 0 contains well-constrained CCD-exposures, while a secondary peak at $\sim -700\mmag$ arises from short/shallow exposures (ZP$_{\FGCM} \sim 28$--$30$~mag) with sparse overlaps in the limited test patch.
The median offset is $-0.6\mmag$, confirming the absolute scale is correct; the 602$\mmag$ RMS is driven entirely by the poorly-constrained tail, which will shrink on the full footprint where the overlap network is denser.

\paragraph{Test 1 --- Photometric repeatability:}
Per-detection scatter for bright stars (20th--40th percentile in magnitude) gives a floor of 8.5~mmag, below the 10~mmag threshold (Figure~\ref{fig:val_repeat}).
The floor rises from $\sim$8~mmag at the bright end to $\sim$13~mmag at $g \sim 20$.

\paragraph{Test 2 --- Anchored comparison:}
The anchored solution achieves 4.8~mmag RMS agreement with FGCM (Figure~\ref{fig:val_dr2}), with the histogram sharply peaked at zero.
This demonstrates the DES anchor propagates correctly through overlaps.

\paragraph{Test 3 --- Legacy Surveys DR10 comparison:}
We cross-match 43,848 stars with LS~DR10 PSF photometry \citep{Dey2019}, which is calibrated to the Pan-STARRS photometric system.
After fitting a linear color term ($23.0 + 3.3 \times (g{-}i)_{\rm LS}\mmag$), the residual scatter is 19.1~mmag RMS (Figure~\ref{fig:val_ls}).
The spatial residual map (Figure~\ref{fig:val_ls}, bottom-left) reveals a $\sim$20--30~mmag offset between the DES (dec~$< -30\degr$) and non-DES (dec~$> -30\degr$) regions --- this is the calibration boundary that the full-sky ubercalibration will reduce.
No significant magnitude-dependent slope is observed (Figure~\ref{fig:val_ls}, bottom-right).

\paragraph{Test 4 --- Gaia DR3 comparison:}
Cross-matching 43,892 stars with Gaia DR3 \citep{GaiaCollaboration2023}, we fit a large color term ($-310.8 + 855.3 \times (\mathrm{BP}{-}\mathrm{RP})\mmag$) as expected from the very different DECam $g$ and Gaia $G$ filter curves.
After color term removal, the residual RMS is 52.6~mmag (Figure~\ref{fig:val_gaia}), just above the 50~mmag threshold.
The spatial residual map shows a pattern similar to Test~3, confirming the boundary origin.

\paragraph{Test 6 --- DES boundary continuity:}
The 110.8~mmag boundary offset between DES (41,261 CCD-exp) and non-DES (5,038 CCD-exp) median solved zero-points (Figure~\ref{fig:val_boundary}) reflects the limited overlap propagation across the boundary in the $10\degr \times 10\degr$ patch.
On the full footprint, with thousands of bridging exposures connecting the two regions, this will improve dramatically.

\begin{figure*}[t]
\centering
\includegraphics[width=\textwidth]{figures/fig10_validation_fgcm.pdf}
\caption{Test~0: Unanchored zero-points vs.\ FGCM for 41,261 DES CCD-exposures. Left: histogram showing the well-constrained peak at zero and a secondary peak at $\sim -700\mmag$ from poorly-constrained exposures. Median offset $= -0.6\mmag$. Right: residual vs.\ ZP$_{\FGCM}$, revealing that the outlier population corresponds to low-ZP (shallow) exposures.}
\label{fig:val_fgcm}
\end{figure*}

\begin{figure*}[t]
\centering
\includegraphics[width=\textwidth]{figures/fig11_validation_repeatability.pdf}
\caption{Test~1: Photometric repeatability for 43,515 stars with $\geq$3 observations. Left: per-detection scatter vs.\ magnitude. Right: median scatter in magnitude bins. The bright-star floor is 8.5~mmag.}
\label{fig:val_repeat}
\end{figure*}

\begin{figure}[t]
\centering
\includegraphics[width=\columnwidth]{figures/fig12_validation_dr2.pdf}
\caption{Test~2: Anchored solution $\Delta$ZP for 41,261 DES CCD-exposures (4.8~mmag RMS, median $= 0.0\mmag$).}
\label{fig:val_dr2}
\end{figure}

\begin{figure*}[t]
\centering
\includegraphics[width=\textwidth]{figures/fig14_validation_ls_dr10.pdf}
\caption{Test~3: Comparison with LS~DR10 (PS1-calibrated) for 43,848 $g$-band stars. Top-left: histogram before (blue, 33.4~mmag RMS) and after (orange, 19.1~mmag RMS) color term removal. Top-right: color term fit vs.\ $(g{-}i)_{\rm LS}$. Bottom-left: spatial residual map after color term, showing the $\sim$20--30~mmag DES boundary at dec~$\approx -31\degr$. Bottom-right: residual vs.\ magnitude (no slope).}
\label{fig:val_ls}
\end{figure*}

\begin{figure*}[t]
\centering
\includegraphics[width=\textwidth]{figures/fig15_validation_gaia.pdf}
\caption{Test~4: Comparison with Gaia DR3 for 43,892 $g$-band stars. Top-left: histogram before (blue, 1001~mmag RMS) and after (orange, 52.6~mmag RMS) color term removal. Top-right: color term fit vs.\ Gaia BP$-$RP (slope $= 855\mmag$/mag, as expected for the very different filter curves). Bottom panels: spatial residual map and magnitude dependence.}
\label{fig:val_gaia}
\end{figure*}

\begin{figure}[t]
\centering
\includegraphics[width=\columnwidth]{figures/fig13_validation_boundary.pdf}
\caption{Test~6: DES boundary continuity. The 110.8~mmag offset between DES (41,261) and non-DES (5,038) CCD-exposure median ZPs is driven by the limited overlap propagation in the $10\degr \times 10\degr$ test patch.}
\label{fig:val_boundary}
\end{figure}

\begin{deluxetable*}{lcccc}
\tablecaption{Phase 6: Validation Test Results (test patch)\label{tab:validation}}
\tablehead{
\colhead{Test} & \colhead{Metric} & \colhead{$g$-band} & \colhead{$r$-band} & \colhead{Status}
}
\startdata
0: FGCM (unanchored) & Median (RMS) & $-0.6$ (602$^{*}$) mmag & 19.2 (853$^{*}$) mmag & expected$^{*}$ \\
1: Repeatability & Floor & 8.5~mmag & 6.8~mmag & PASS \\
2: Anchored vs FGCM & RMS & 4.8~mmag & 6.0~mmag & PASS \\
3: LS DR10 (PS1) & RMS after CT & 19.1~mmag & 22.3~mmag & PASS \\
4: Gaia DR3 & RMS after CT & 52.6~mmag & 109.4~mmag & marginal / FAIL \\
5: Stellar locus & --- & --- & --- & SKIP \\
6: DES boundary & Offset & 110.8~mmag & 208.3~mmag & expected$^{*}$ \\
\enddata
\tablecomments{$^{*}$Expected to improve on full footprint: the unanchored RMS is dominated by poorly-constrained exposures in the sparse overlap network, and the boundary offset requires dense cross-boundary overlaps. The $r$-band boundary test has only 168 non-DES CCD-exposures (vs.\ 5,038 in $g$).}
\end{deluxetable*}

\subsection{$r$-Band Validation}

The $r$-band pipeline ran on the same $10\degr \times 10\degr$ test patch.
Results are consistent with $g$-band (Table~\ref{tab:validation}).
The anchored solution achieves 6.0~mmag RMS agreement with FGCM (vs.\ 4.8 in $g$), and the repeatability floor is 6.8~mmag (vs.\ 8.5 in $g$), both comfortably below the 10~mmag target.

The LS~DR10 comparison (Figure~\ref{fig:val_ls_r}) shows a larger color term offset (135.6~mmag) than $g$-band (23.0~mmag), reflecting a larger DECam $r$ vs.\ PS1 $r$ filter difference.
After color term removal, the scatter is 22.3~mmag RMS, comparable to $g$.
The spatial residual map again shows the dec~$\approx -31\degr$ boundary.

The Gaia comparison (Figure~\ref{fig:val_gaia_r}) shows 109~mmag RMS after a linear color term, significantly worse than $g$-band (53~mmag).
The color term relationship between DECam $r$ and Gaia $G$ is clearly nonlinear (Figure~\ref{fig:val_gaia_r}, top-right), and a quadratic fit would substantially improve the residuals.

The DES boundary test yields 208~mmag offset, but with only 168 non-DES CCD-exposures in $r$-band (vs.\ 5,038 in $g$), this comparison is not meaningful.

\begin{figure*}[t]
\centering
\includegraphics[width=\textwidth]{figures/validation_ls_dr10_r.pdf}
\caption{Test~3 ($r$-band): Comparison with LS~DR10 (PS1-calibrated) for 68,620 stars. The color term offset (135.6~mmag) is larger than in $g$-band (23.0~mmag). After removal, the residual scatter is 22.3~mmag RMS. The spatial residual map (bottom-left) shows the same dec~$\approx -31\degr$ boundary as $g$-band.}
\label{fig:val_ls_r}
\end{figure*}

\begin{figure*}[t]
\centering
\includegraphics[width=\textwidth]{figures/validation_gaia_r.pdf}
\caption{Test~4 ($r$-band): Comparison with Gaia DR3 for 68,196 stars. The DECam $r$ -- Gaia $G$ color term is clearly nonlinear (top-right), leading to 109~mmag RMS after a linear fit. A quadratic color term would improve this comparison.}
\label{fig:val_gaia_r}
\end{figure*}

\subsection{Pipeline Summary}

Table~\ref{tab:summary} provides a comprehensive summary of all validation metrics across the pipeline.

\begin{deluxetable*}{llrl}
\tablecaption{Validation Gate Summary ($g$-band, $10\degr \times 10\degr$ test patch)\label{tab:summary}}
\tablehead{
\colhead{Phase} & \colhead{Gate} & \colhead{Value} & \colhead{Status}
}
\startdata
0 & Stars in test patch & $\sim$103,000 & \checkmark \\
0 & Detections in test patch & 1,004,190 & \checkmark \\
0 & Aperture & 4\arcsec~diameter (mag\_aper4) & \checkmark \\
0 & Max detections per star & $\leq 25$ & \checkmark \\
\hline
2 & CG convergence (unanchored) & 1,120 iterations & \checkmark \\
2 & CG convergence (anchored) & 94 iterations & \checkmark \\
2 & Relative residual & $< 10^{-5}$ & \checkmark \\
\hline
3 & Residual RMS improvement & $124.4 \to 11.0$~mmag & \checkmark \\
3 & Anchored DES diff RMS & 4.8~mmag & \checkmark \\
3 & Stars flagged & 57,766 & \checkmark \\
\hline
4 & Mean correction amplitude & 5.7~mmag & \checkmark \\
4 & CCD groups fitted & 123 & \checkmark \\
\hline
5 & Stars in catalog & 45,087 & \checkmark \\
5 & Mag range & [17.0, 20.0] & \checkmark \\
5 & NaN/Inf magnitudes & 0 & \checkmark \\
\hline
6 & Repeatability floor & 8.5~mmag ($<$10) & \checkmark \\
6 & Anchored DES RMS & 4.8~mmag ($<$15) & \checkmark \\
6 & LS DR10 RMS (after CT) & 19.1~mmag & \checkmark \\
6 & Gaia RMS (after CT) & 52.6~mmag & marginal \\
6 & Boundary offset$^{*}$ & 110.8~mmag & \checkmark \\
\hline
All & Unit tests & 65/65 pass & \checkmark \\
\enddata
\tablecomments{$^{*}$Expected to improve on full footprint (limited cross-boundary overlaps in test patch).}
\end{deluxetable*}

%======================================================================
\section{Discussion} \label{sec:discussion}

The test patch results demonstrate that the pipeline produces physically meaningful ubercalibrated magnitudes and achieves $\lesssim 5\mmag$ agreement with DES FGCM in anchored mode.
Several points merit discussion:

\paragraph{DES boundary and anchored vs.\ unanchored modes.}
The most striking result is the contrast between the two solve modes at the dec~$= -30\degr$ boundary.
In \emph{unanchored} mode, the DES--non-DES median ZP offset is $0.0\mmag$: the overlaps naturally equalize the two regions.
In \emph{anchored} mode, a 110.8~mmag boundary appears because pinning DES CCD-exposures to their FGCM values with anchor weight $\alpha = 10^6$ introduces tension with the overlap constraints, particularly for poorly-constrained exposures (which have $\sim$600~mmag scatter in the unanchored DES--FGCM comparison).
This tension propagates to non-DES neighbors that lack the strong anchor constraint.
On the full footprint, with thousands of bridging exposures and a much denser overlap network across the boundary, this tension will be better distributed and the boundary offset should decrease.

\paragraph{Unanchored DES comparison.}
The 602~mmag RMS in the unanchored DES--FGCM comparison (Test~0) is driven by a bimodal distribution: a well-constrained population near zero and a secondary population of poorly-constrained exposures at $\sim -700\mmag$.
The latter corresponds to shallow exposures ($\ZP_{\FGCM} \sim 28$--$30$~mag) with sparse overlaps in the limited test patch.
The median offset is $-0.6\mmag$, confirming the absolute scale is correct.

\paragraph{External comparisons.}
The LS~DR10 comparison (Test~3) yields 19.1~mmag RMS after a small color term ($23.0 + 3.3 \times (g{-}i)\mmag$), consistent with the expected DECam $g$ vs.\ PS1 $g$ filter difference.
The spatial residual map clearly shows the dec~$\approx -31\degr$ boundary at $\sim$20--30~mmag amplitude, providing a direct visualization of the calibration discontinuity.

\paragraph{Star flat amplitudes.}
The mean star flat correction of 5.7~mmag is consistent with literature values for DECam ($\sim$5~mmag; \citealt{Bernstein2017}).

%======================================================================
\section{Summary} \label{sec:summary}

We have implemented and validated a photometric ubercalibration pipeline for the \delve~survey.
The key results from the $g$-band $10\degr \times 10\degr$ test patch (40 HEALPix pixels, 73,369 CCD-exposures) are:

\begin{enumerate}
    \item The CG sparse solver converges in both anchored (94 iterations) and unanchored (1,120 iterations) modes with relative residual $< 10^{-5}$.
    \item Iterative outlier rejection reduces the residual RMS from 124.4 to 11.0~mmag and the anchored DES--FGCM comparison to 4.8~mmag RMS.
    \item Per-CCD star flat corrections have a mean amplitude of 5.7~mmag, consistent with DECam literature values.
    \item The final catalog of 45,087 stars has physical magnitudes ($17 < g < 20$) with a bright-star repeatability floor of 8.5~mmag.
    \item External comparison with LS~DR10 (PS1-calibrated) gives 19.1~mmag RMS after color term removal, with no magnitude-dependent slope.
    \item The unanchored solve shows \emph{zero} DES/non-DES boundary offset, demonstrating that the overlaps naturally connect the two regions. The 111~mmag boundary in anchored mode reflects anchor-induced tension in the sparse test-patch overlap network.
    \item All 65 unit tests pass across 6 pipeline phases.
\end{enumerate}

The pipeline is currently processing all $\sim$9,664 pixels across $griz$ ($r$-band Phase~0 in progress).
We expect the final calibration to achieve 5--10~mmag uniformity across the full $\sim$17,000~deg$^2$ footprint once the overlap graph is fully connected and the anchor tension is distributed over a much denser network of bridging exposures.

\acknowledgments

This work was developed using Claude Code by Anthropic.
Based on observations at Cerro Tololo Inter-American Observatory, a program of NSF NOIRLab.
This research uses services or data provided by the Astro Data Lab, which is part of the Community Science and Data Center (CSDC) Program of NSF NOIRLab.

\bibliography{references}

\end{document}
